\documentclass[12pt,letterpaper]{article}

% Librerías a utilizar
\usepackage[utf8]{inputenc}	% Codificación 
\usepackage[style=apa, backend=biber, language=spanish]{biblatex}
\usepackage{graphicx}		% Igenes
\usepackage{indentfirst}		% Sangría
\usepackage{amsmath, amsfonts, amssymb}	% Figuras matemáticas
\usepackage{xurl}    % URL
\usepackage{hyperref}
\usepackage{csquotes}
\usepackage{booktabs}
\usepackage{placeins}
\usepackage{tabularx}
\usepackage[spanish,es-tabla]{babel}	% Idioma
\addbibresource{bibliografia.bib}

% Fuente Charter
% Si prefieren utilizar la original, pueden comentar la siguiente línea
\usepackage{charter}

\usepackage[left=3cm,right=2cm,top=2cm,bottom=3cm]{geometry}
\usepackage{multirow}

\setlength{\parindent}{2cm}	% Sangría en los párrafos
\renewcommand{\baselinestretch}{1.5}	% Interlineado

% Renombrar ciertos títulos del texto
\renewcommand\spanishcontentsname{Tabla de contenidos}
\renewcommand\spanishrefname{Bibliografía}

% Inicio del documento
\begin{document}

%%%%%%%%% PORTADA %%%%%%%%%

\newpage
\vspace*{-.5cm}
% Logo institucional
\begin{picture}(18,4)(0,30)
	\put(350,-20){\includegraphics[scale=0.08]{./images/USACH P2.png}}
\end{picture}

\sloppy
\thispagestyle{empty}
\vspace*{-1.6cm}

% Datos institucionales
\begin{center}
	{\bf \mbox{\large UNIVERSIDAD DE SANTIAGO DE CHILE}}\\
	{\bf \mbox{FACULTAD DE INGENIER\'IA}}\\
	{\bf \mbox{DEPARTAMENTO DE INGENIER\'IA INFORM\'ATICA}}\\
\end{center}

	\vspace{5cm}
 
	%Título del trabajo
	\begin{center}
	\Large
		\textbf{Laboratorio 1}\\
        \normalsize{\textbf{Subtítulo}}
	\end{center}
	
	% Datos personales
	\vspace*{6.25cm}
	\begin{flushright}
		\begin{tabular}[t]{l l}
			Integrantes: &Integrante 1\\
                         &Integrante 2\\
			Curso: &Análisis de Datos\\
			Profesor: &Dr. Max Chacón\\
			Ayudante: &Marcelo Álvarez

		\end{tabular}
	\end{flushright}
	\begin{center}
		\vspace{1.5cm}
		% Fecha
		\today
	\end{center}

\newpage
\tableofcontents
\thispagestyle{empty}

\newpage
\renewcommand{\thepage}{\arabic{page}}
\setcounter{page}{1}

% Capítulos agregados 
\section{Sección 1}

\section{Sección 2}

\section{Sección 3}
\section{Sección 4}


\clearpage
\addcontentsline{toc}{section}{Bibliografía}
\printbibliography


\end{document}